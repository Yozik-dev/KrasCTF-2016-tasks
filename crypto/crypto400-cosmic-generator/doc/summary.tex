\documentclass{article}

\usepackage[utf8]{inputenc}
\usepackage[russian]{babel}
\usepackage{fancyvrb}
\usepackage{multicol}
\usepackage{listings}
\usepackage{parcolumns}
\usepackage{verbdef}
\usepackage{courier}
\usepackage{indentfirst}
\usepackage{framed}


\begin{document}
	
	\renewcommand{\contentsname}{Contents}
	
	\section{Description}
	
	\tt Title: Cosmic generator \\
	
	\tt Category: Crypto \\
	
	\tt Value: 400
	
	\begin{framed}
		Дан файл spacesecrets.jpg, содержащий зашифрованную информацию. Также, дан файл encode.py, осуществляющий шифрование и spacenoize.pcap с дампом трафика.
	\end{framed}
	
	\section{Legend}
	
	Космос пронизывают бесчисленные множества радиоволн, некоторые из которых являются просто случайным шумом, а некоторые специально сгенерированным.
	
	\section{Flag}
	
	\tt Flag regex: 
	\framebox[.7\textwidth\hfill]{simple text} \\
	
	\tt Flag: 
	\framebox[.7\textwidth\hfill]{дажешумсодержитсмысл}
	
	\section{Solution}
	
	В питоновском скрипте реализован стандартный алгоритм шифрования при помощи XOR с псевдослучайной последовательностью, сгенерированной линейным сдвиговым регистром с обратной связью. В дампе трафика можно найти ICMP пакеты и собрать из их содержимого последовательность (коды 30 и 31 - это ASCII коды нуля и единицы). Далее следует догадаться, что поксорив ее с зашифрованной картинкой - получается валидная сигнатура. Однако, последовательность только 120 байт, а нужно для картинки гораздо больше. Применяем алгоритм Берлекемпа-Месси и получаем результат. Подробнее в файле solver.py. Возможны и иные варианты ввиду того, что seed и polynom имеют небольшые значения.
	
	\section{Hints}
	
	\begin{enumerate}
		\item Космический шум, всем известно, передается пингами по 10 значений в каждом.
		\item Это довольно известный алгоритм генерации гаммы.
		\item иссэМ — апмэкелреБ мтироглА.
		\item Алгоритм Берлекэмпа — Мэсси.
		\item Если совсем не идет - seed 3 бита, polynom 12 бит.
	\end{enumerate}
	
	\section{Discussion}
	
	Тут и проще и сложнее некуда. Разве что можно PCAP похитрее закодировать.
	
	\section{Setup}
	
	Выдать командам файлы censored.jpg, encoding\_script.py, spacenoize.pcap.
	
\end{document}