\documentclass{article}

\usepackage[utf8]{inputenc}
\usepackage[russian]{babel}
\usepackage{fancyvrb}
\usepackage{multicol}
\usepackage{listings}
\usepackage{parcolumns}
\usepackage{verbdef}
\usepackage{courier}
\usepackage{indentfirst}
\usepackage{framed}


\begin{document}
	
\renewcommand{\contentsname}{Contents}

\section{Description}

\tt Title: Квадратная прецессия \\

\tt Category: Crypto \\

\tt Value: 100

\begin{framed}
	Даны три шифротекста (ciphertext1.txt, ciphertext2.txt, ciphertext3.txt) и два открытых текста (plaintext1, plaintext2), зашифрованные одним и тем же шифром. Требуется расшифровать третий текст и найти в нем флаг.
\end{framed}

\section{Legend}

\begin{framed}
	Наши разведчики перехватили важную переписку с серверов противника. К сожалению, перехватить в открытом виде удалось только незначительные сообщения. Ваша цель - расшифровать все.
\end{framed}

\section{Flag}

\tt Flag regex: 
\framebox[.7\textwidth\hfill]{simple text} \\

\tt Flag: 
\framebox[.7\textwidth\hfill]{gravitationfollowsspacecreatures}

\section{Solution}

Заметим, что частотность символов не изменяется - это шифр перестановки. Длина сообщений является точным квадратом, намекает на квадратную таблицу. Запишем шифротекст в таблицу по строкам и увидим, что открытый текст вырисовывается зигзагом. Восстанавливаем исходный текст.
Более подробно можно посмотреть в файле task.py и функции test().

\section{Hints}

\begin{enumerate}
\item Заметьте, что частотность символов не изменяется.
\item Измерьте длину сообщений.
\item Это перестановка внутри квадратной таблицы.
\item Зигзаг же с левого верхнего угла!
\end{enumerate}

\section{Discussion}

Возможно усложнить задачу путем внесения дополнительных перемешиваний. Можно упростить задачу путем уменьшения третьего текста, чтобы им не писать программу, а делать задание вручную.

\section{Setup}

Выдать командам архив с файлами ciphertexts/*.txt и plaintexts/\{1.txt, 2.txt\}.

\end{document}