\documentclass{article}

\usepackage[utf8]{inputenc}
\usepackage[russian]{babel}
\usepackage{fancyvrb}
\usepackage{multicol}
\usepackage{listings}
\usepackage{parcolumns}
\usepackage{verbdef}
\usepackage{courier}
\usepackage{indentfirst}
\usepackage{framed}


\begin{document}
	
	\renewcommand{\contentsname}{Contents}
	
	\section{Description}
	
	\tt Title: Cats in space \\
	
	\tt Category: Crypto \\
	
	\tt Value: 200
	
	\begin{framed}
		Даны файлы letter.pdf и letter.txt, содержащие в себе кучу картинок котиков и планеток. Требуется извлечь текст.
	\end{framed}
	
	\section{Legend}
	
	Вы хорошо поработали над предыдущим заданием. Однако, расшифрованы не все письма! Парочка завалялось в мусорном контейнере. Мы не смогли их расшифровать т.к. залипли на содержание. Удачи.
	
	\section{Flag}
	
	\tt Flag regex: 
	\framebox[.7\textwidth\hfill]{simple text} \\
	
	\tt Flag: 
	\framebox[.7\textwidth\hfill]{флагкосмосфлаг}
	
	\section{Solution}
	
	Первый шифротекст зашифрован шифром Цезаря со сдвигом на 16. Теперь посмотрим на второй. Если посчитать - видов котиков чуть больше 33, что намекает на шифр замены. Планеты стоят чаще, а значит кодируют пробелы и прочие спец. символы. Извлечь картинки из PDF можно как вручную, так и простым скриптом на питоне (там довольно простая структура, если открыть в блокноте). Далее воспользуемся либо спец. софтом на базе частотного анализа, либо намеком в первом шифротексте на то, что текст начинается с фразы "письмо от ...".
	
	\section{Hints}
	
	\begin{enumerate}
		\item Spaces are spaces.
		\item Подумайте внимательно какая у письма может быть структура на основании того, что вы знаете о письмах.
		\item Котики кодируют буквы, а планетки - пробелы, табы, переносы строк и т.д.
		\item Сообщение начинается с фразы "письмо от".
	\end{enumerate}
	
	\section{Discussion}
	
	Можно упростить задачу, изъяв из нее спецсимволы и добавив в текст наличие цифр так, чтобы получилось ровно 43 элемента алфавита. Планетами кодировать только пробелы и переносы строк.
	Усложнять задачу вряд-ли потребуется.
	
	\section{Setup}
	
	Выдать командам файлы letter2.pdf и letter1.txt
	
\end{document}