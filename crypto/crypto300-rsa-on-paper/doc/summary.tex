\documentclass{article}

\usepackage[utf8]{inputenc}
\usepackage[russian]{babel}
\usepackage{fancyvrb}
\usepackage{multicol}
\usepackage{listings}
\usepackage{parcolumns}
\usepackage{verbdef}
\usepackage{courier}
\usepackage{indentfirst}
\usepackage{framed}


\begin{document}
	
\renewcommand{\contentsname}{Contents}

\section{Description}

\tt Title: RSA на бумаге \\

\tt Category: Crypto \\

\tt Value: 300

\begin{framed}
	Дан файл, зашифрованный с помощью RSA и публичный ключ, с помощью которого производилось шифрование. В комплекте идет фотография, распечатанного на бумаге приватного RSA ключа, который частично поврежден.
\end{framed}

\section{Legend}

\begin{framed}
	Инопланетный разум использует исключительно RSA для шифрования своей переписки. Нашему агенту удалось сфотографировать ключ через иллюминатор космического корабля.
\end{framed}

\section{Flag}

\tt Flag regex: 
\framebox[.7\textwidth\hfill]{simple text} \\

\tt Flag: 
\framebox[.7\textwidth\hfill]{whyplutoisnotaplanetanymore}

\section{Solution}

Применяем стандартную атаку ферма на RSA. Придется постараться, чтобы распознать все символы ключа автоматически т.к. изображение достаточно большое. По ключу видно, что в нем много нулей, и его p и q вероятно очень близко друг к другу расположены. Поврежденный на картинке символ придется перебрать.
Для перевода из PEM формата ключа можно воспользоваться библиотекой pycrypto или openssl.

\section{Hints}

\begin{enumerate}
\item Присмотритесь к ключу, почему он такой неслучайный.
\item PEM формат, если что.
\item Атака как по википедии.
\end{enumerate}

\section{Discussion}

Никаких особых дополнений.

\section{Setup}

Выдать командам архив с файлами encrypted.enc, photo.jpg.

\end{document}